\documentclass[a4paper]{article}
\usepackage[utf8]{inputenc}
\usepackage[danish]{babel}
\usepackage[T1]{fontenc}
\usepackage{hyperref}
\usepackage[margin=2.5cm]{geometry}
\usepackage{amsmath}

\usepackage{graphicx}
\graphicspath{/}

\title{POP assignment 2i}

\author{Mikkel \emph{Luc} Carlsen}

\begin{document}

    \maketitle 

    \section*{Task \normalfont{2i0}}

    The three different methods that can be used to run an Fsharp program is(assuming that you have navigated into the folder 
    that you are working in):

    \begin{itemize}
        \item 1) Write ''fsharpi'' in the terminal to launch Fsharp in the terminal and then write the code directly into the terminal
        \item 2) Write ''fsharpi \emph{Fsharp file name}'' in the terminal, E.g. ''fsharpi 2i1.fsx'', where 2i1.fsx is the name of my 
        Fsharp file
        \item 3) Write ''fsharpc \emph{Fsharp file name}'' in the terminal. After the command has loaded type ''mono 
        \emph{Fsharp file name, but now ending with \textbf{.exe} instead of \textbf{.fsx}}'' into the terminal E.g. 
        my case would be ''fsharpc 2i1.fsx'' and the ''mono 2i1.exe''
    \end{itemize}

    The first listed method of launching an Fsharp program is advantageous, when testing small fragments of code. 
    This is because ''fsharpi'' launches Fsharp directly in the terminal, I.e it executes the written code as soon as you type '';;'' 
    and press enter, instead of having to compile it first then execute it. On the other hand it's disadvantageous when creating
    a program, because it's necessary to save the program as a file, I.e you are forced to write the code in a file and then 
    either rewrite it in the terminal or copy paste all of the code into the terminal.\\

    The second listed method of launching an Fsharp program grants the advantage of being able to write bigger piles of code 
    in a file and then test it directly - noticable that its advantage is exactly what was the first methods disadvantage. 
    So if the command ''fsharpi \emph{Fsharp file name}'' were to be executed, it would execute the program without having to 
    copy paste it. The disadvantage of this method is that it has a overall slow running time, but it's slighty faster than the
    third way if the purpose is to run the code only once or if it's necessary to recompile everytime before running.\\

    The third listed method is advantageous when running the program multiple times without having to recompile. This is due to 
    the fact that it takes almost no time to run the program with mono when it's already compiled I.e. when the program is 
    already compiled it's possible to just use the mono command without reusing the fsharpc command. But even so, it's slower
    to use the fsharpc and mono command together, than it's to use the fsharpi \emph{Fsharp file name} command. Therefore it's only
    advantageous when the program is complete and does not require further changes.
    
\newpage

    \section*{Task \normalfont{2i1}}

    The goal of the second task was to create an expression in Fsharp, that could extract the words ''hello'' and ''world'' 
    from a string saying ''hello world''.\\
    The way i solved this task was by writing the following in Fsharp:\\
\begin{verbatim}
    let a = ''Hello world''\\
    let b = a.[0..4]\\
    let c = a.[6..10] \\
    printfn ''‰A'' b
    \\
    printfn ''‰A'' c
    \\
\end{verbatim}
    What the code does is that it defines the variable \emph{a} as "hello world"
    Then it defines the variable \emph{b} as the lenght of \emph{a} from 0-4, I.e. it defines \emph{b} as the values in a[0,1,2,3,4]
    The same goes for \emph{c}, but for the values a[6,7,8,9,10]
    afterwards it prints \emph{b} and then \emph{c}
    \section*{Task \normalfont{2i2}}

    For task 2i2 i completed it with pen and paper, but i have added the table to assigntment as seen below.

    \begin{table}[h!]
        \begin{tabular}{lllll}
        Decimal & binary & Hexadecimal & Octal &  \\
        10      & 1010   & A           & 12    &  \\
        21      & 10101  & 15          & 25    &  \\
        47      & 101111 & 2f          & 57    &  \\
        59      & 111011 & 3b          & 73    & 
        \end{tabular}
        \end{table}
    The way i completed the table was by filling out the decimal row first.
    so step by step, i started out with converting the binary number 10101$_2$ to a Decimal. This was done by saying 
    $d_n*2^n+d_{n-1}*2^{n-1}$ e.g. if the binary number is $10101_2$ then it would be $1*2^4+0*2^3+1*2^2+0*2^1+1*2^0 --> 16+0+4+0+1=21_{10}$.
    After that i converted the hexadecimal $2f_{16}$ to decimal by using the power $16^n$ method starting from the right. e.g. $2f_{16}$ would be converted by saying $16^0*f_{16}+16^1*2-->1*15+32=47_{10}$.
    Then i converted octal to decimal taking each digit in the octal number from the left to right and multiplying it with its corresponding $8^n$. e.g. $73_8-->7*8^1+3*8^0=56+3=59_{10}$
    Now that i had filled the decimal row, i used it to convert decimal to the remaining missing values of binary, Hexadecimal and octal numbers.\\

    Decimal --> binaryinary: Done with the integer divide and remainding method e.g. \\
    $10/2=5, 10rem2=0 \\
     5/2=2, 5rem2=1 \\ 
     2/2=1, 2rem2=0 \\ 
     1/2=0, 1rem2=1$. \\
    From here you can read the remainding numbers from bottom to top so it says that the decimal $10_{10}$ is equal to the binary $1010_2$.\\

    Decimal --> hexadecimal: Done by integer dividing the decimal with 16, then add that to the result and take the remaining numbers from the devision and convert to hexadecimal. E.g. $59_{10}/16=3, rem=11$ which is equal to $3b_{16}$.\\

    Decimal --> octal: Was done with a homemade version. For every 10th decimal i would say $12*n$, where n is the amount of 10ths 
    in the decimal number. Then if the number i ended up with would have the first digit as either 8 or 9, then if the single 
    digit was 8 i would round up to the next 10th else if it was 9 i would round up to the next 10th+1. after that that i would 
    finally add the remaining single digits from the decimal. e.g. if the decimal is $47_{10}$ i would notice that the 10th is 40,
    I.e. that n=4. So i would say $12*4=48$ and then see that the first digit is 8, which means i have to round up to the 
    nearest 10th, which is 50. Now i add the remaining single digits from the decimal so that i get my octal number $50+7=57_8$

    \end{document}