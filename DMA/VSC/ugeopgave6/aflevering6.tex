\documentclass[12pt,a4paper]{article}

% Packages
%\usepackage[english,danish]{babel}
%\usepackage[applemac]{inputenc}
\usepackage{amsmath,amscd}
\usepackage{amssymb}
\usepackage{amsthm}
\usepackage{enumerate}
\usepackage{graphicx}                    
\usepackage{framed}  

\theoremstyle{plain}
\newtheorem{thm}{SŸtning}
\newtheorem{prop}[thm]{Proposition}
\newtheorem{lem}[thm]{Lemma}
\newtheorem{cor}[thm]{Korollar}
\newtheorem{conj}[thm]{Formodning}
\theoremstyle{definition}
\newtheorem{exercise}{Opgave}
\newtheorem{definition}[thm]{Definition}
\newtheorem{prob}[thm]{Problem}
\newtheorem{remark}[thm]{BemŸrkning}
\newtheorem{example}[thm]{Eksempel}


% Blackboard bold
\newcommand{\NN}{\mathbb{N}}
\newcommand{\ZZ}{\mathbb{Z}}
\newcommand{\QQ}{\mathbb{Q}}
\newcommand{\RR}{\mathbb{R}}
\newcommand{\CC}{\mathbb{C}}
\newcommand{\non}{\sim \hspace{-1mm}}


\title{Assigment 6 DMA}
\author{Mikkel Luc Carlsen}


\begin{document}

\maketitle

\begin{itemize}
    \item[Part 1] 
    The statement "f(x) is O(g(x))" can be expressed using logical operators as the following:
    \begin{equation}
        \exists c>0 \; \exists x_0\in \RR^+ \; \forall x\ge x_0 \; f(x) \le cg(x)
    \end{equation}
    \begin{enumerate}
    \item[(a)]
    The negation of this expression can simply be expressed as:
    \begin{equation}
        \sim [\exists c>0 \; \exists x_0\in \RR^+ \; \forall x\ge x_0 \; f(x) \le cg(x)]
    \end{equation}  
    I will now use theorem 3 from KBR chapter 2.2 in order to simplyfy it. Now the equation will look like this:
    \begin{equation}
        \forall c>0 \; \forall x_0\in \RR^+ \; \exists x\ge x_0 \; \sim[f(x) \le cg(x)]
    \end{equation}
    now all thats left is to remove the $\sim$ notation, this is done as follows:
    \begin{equation}
        \forall c>0 \; \forall x_0\in \RR^+ \; \exists x\ge x_0 \; f(x) > cg(x)
    \end{equation}
    and now we have the negation of the former expression.

    \item[(b)]
    To express this very shortly and directly in danish one could say: "For enhver konstant $c>0$ og for ethvert $x_0$ i mængden $\RR^+$, 
    findes der et $x\ge x_0$, sådan at $f(x) > cg(x)$.
\end{enumerate}\

\item[Part 2]
    The statement "If r is an irrational number, then $r^{1/5}$ is an irrational number" can be proven by proving the contrapositive statement. 

    \textbf{Impliation $p\Rightarrow q$}: If r is an irrational number, then $r^{1/5}$ is an irrational number.
    \newline \textbf{Contrapositive $\sim q \Rightarrow \sim p$}: If $r^{1/5}$ is a rational number, then r is a rational number.

    Now if we assume that $r^{1/5}$ is a rational number, then it means that we can write it as $a/b$ so $r^{1/5}=a/b$ for some intergers \textit{a} as a 
    numerator and \textit{b} as a non-zero denominator.
    So from this we can also conclude that:
    \begin{equation}
        (r^{1/5})^5=(a/b)^5 \leftrightarrow r=a^5/b^5
    \end{equation}
    Now from this we get an expression for r, which is $a^5/b^5$ where both \textit{a} and \textit{b} are intergers meaning that $a^5$ and $b^5$ is also intergers and therefore we might aswell express it as $r=p/q$, where $p=a^5$ and $q=b^5$ meaning that r is rational number and we have therefor proved the contrapositive of our original statement.
    
\item[Part 3]
    By the definition of Big-O then $4^n$ is not $O(2^n)$.\\
    This can be proven by doing a proof of contradiction.

    The definition of big-O is as follows: "We say that $f(x)$ is $O(g(x))$ if there exists a constant $c > 0$ and $x_0$ such that $f(x)\leq cg(x)$ for all $x\geq x_0$.

    Now because this is a proof of contradiction, then we want to try to proof the negation of our original statement and meet a contradiction for this, which we can use to conclude that our original statement is true since the negation then is false.
    Therefore we wish to prove the negation of \textit{"$4^n$ is not $O(2^n)$"} which is \textit{"$4^n$ is $O(2^n)$"} meaning that we can find a constant $c > 0$ and $n_0$ such that $4^n\leq c*2^n$ for all $n\geq n_0$.
    So we have:
    \begin{equation}
        4^n\leq c*2^n
    \end{equation}
    By dividing each side of the equation with $2^n$ then we get
    \begin{equation}
        2^n\leq c
    \end{equation}
    Now this expression has to count for all $n \geq n_0$, which is a contradiction because no matter which c we chose, there will eventually be an $n \geq n_0$ that will overcome this expression so that $2^n\geq c$ will count instead and therefor $4^n$ is not O$(2^n)$

\item[Part 4]
\begin{enumerate}
    \item[(a)]
    An equivalent expression for ($\sim$P) using $\odot$ is showed in the following table:
    
        \begin{tabular}{|c|c|lcl|c|lcl|}
            \multicolumn{1}{|l|}{P} & \multicolumn{1}{l|}{Q} & P & \multicolumn{1}{l}{$\odot$} & Q & \multicolumn{1}{l|}{($\sim$P)} & P & \multicolumn{1}{l}{$\odot$} & P \\ \hline
            T                       & T                      &   & F                        &   & F                          &   & F                        &   \\
            T                       & F                      &   & T                        &   & F                          &   & F                        &   \\
            F                       & T                      &   & T                        &   & T                          &   & T                        &   \\
            F                       & F                      &   & T                        &   & T                          &   & T                        &  
        \end{tabular} \\
    
    $\odot$ can be considered the following way to support the result of the truth table above:\\
        T $\odot$ T = F \\
        T $\odot$ F = T \\
        F $\odot$ T = T \\
        F $\odot$ F = T

    \item[(b)]
    An equivalent expression for each of the two expressions can be expressed as following:

    \begin{itemize}
        \item[(i)]
        $P \lor Q \leftrightarrow (\sim P)\odot (\sim Q)$ \\
    \end{itemize} 

        \begin{tabular}{|c|c|lcl|ccc|c|c|lcl|}
                \multicolumn{1}{|l|}{P} & \multicolumn{1}{l|}{Q} & P & \multicolumn{1}{l}{$\lor$} & Q & \multicolumn{1}{l}{P} & \multicolumn{1}{l}{$\odot$} & \multicolumn{1}{l|}{Q} & \multicolumn{1}{l|}{($\sim$P)} & \multicolumn{1}{l|}{($\sim$Q)} & ($\sim$P) & \multicolumn{1}{l}{$\odot$} & ($\sim$Q) \\ \hline
                T                       & T                      &   & T                      &   &                       & F                        &                        & F                          & F                          &       & T                        &       \\
                T                       & F                      &   & T                      &   &                       & T                        &                        & F                          & T                          &       & T                        &       \\
                F                       & T                      &   & T                      &   &                       & T                        &                        & T                          & F                          &       & T                        &       \\
                F                       & F                      &   & F                      &   &                       & T                        &                        & T                          & T                          &       & F                        &      
        \end{tabular} \\

    \begin{itemize}
        \item[ (ii)] 
        $P$ \texttt{xor} $Q$ $\leftrightarrow$ $\sim((P\odot Q)\odot ((\sim P)\odot(\sim Q)))$\\
    \end{itemize}

    \begin{tabular}{|c|c|lcl|cll|}
        \multicolumn{1}{|l|}{P} & \multicolumn{1}{l|}{Q} & P & \multicolumn{1}{l}{\texttt{xor}} & Q & $\sim((P\odot Q)$ & $\odot$ & $((\sim P)\odot(\sim Q)))$ \\ \hline
        T                       & T                      &   & F                       &   &                & F    &                       \\
        T                       & F                      &   & T                       &   &                & T    &                       \\
        F                       & T                      &   & T                       &   &                & T    &                       \\
        F                       & F                      &   & F                       &   &                & F    &                      
        \end{tabular} \\

    The former truth tables can be used to see that this expression holds. 

      
\end{enumerate}   

\end{itemize}


\end{document}

